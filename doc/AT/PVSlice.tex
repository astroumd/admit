% in RST now, don't edit here

\subsection{PVSlice\_AT}

\subsubsection{Description}

PVSlice\_AT creates a position-velocity slice through a spectral
window cube, which
should be a representative version showing all the spectral lines.

\subsubsection{Use Case}

A position-velocity slice is a great way to visualize all the emission
in a spectral window cube, next to CubeSum.

\subsubsection{Input BDPs:}


\begin{description}

\item[SpwCube] Spectral Window Cube to take the slice through.  We normally
mean this to be a Position-Position-Velocity (or Frequency) cube.

\item[CubeSum]  Optional. One of CubeSum or PeakPointPlot can be used to
estimate the best slit.
It will use a moment of inertia analysis to decipher the best line 
in RA-DEC for the slice.

\item[PeakPointPlot]  Optional.


\end{description}


\subsubsection{Input keywords}

\begin{description}

\item[center]
Center and Position angle of the line in RA/DEC.  E.g.   129,129,30
Optional length?  Else full slice through cube is taken.\\
*** NEED TO DESCRIBE HOW POSITION ANGLE IS DEFINED.  EAST OF NORTH? ***

\item[line]
Begin and End points of the line in RA/DEC.  E.g.   30,30,140,140\\
*** WHAT ARE THE UNITS? COULD THIS BE DONE WITH A REGION KEYWORD? ***

\item[major]
Place the slit along the major axis?  Optionally the minor axis slit can
be taken. For rotation flow the major axis makes more  sense, for outflows
the minor axis makes more sense.  Default: True.

\item[width]
Width of the slit. By default sampling through the cube is
done, which equals a zero thickness, but a finite number of
pixels can be choosen as well, in which case the signal within
that width is averaged.
Default: 0.0

\item[minval]
Minimum intensity value below which no data from CubeSum or PeakPointPlot are
used to determine the best slit.

\item[gamma] 
The factor by which intensities are weighed (intensity**gamma) to compute
the moments of inertia from which the slit line is computed.
Default: 1




\end{description}

\subsubsection{Output BDPs}

\begin{description}

\item[PVSlice\_BDP] 

\end{description}


\subsubsection{Procedure}

Either a specific line is given manually, or it can be derived from a reference map (or table).
By this it computes a (intensity weighted) moment of inertia, which then defines a major and minor
axis. Spatial sampling on output is the same as on input (this means if there ever
is a map with unequal sampling in RA and DEC, there is an issue, WSRT data?).


\subsubsection{CASA tasks used}

\begin{description}

\item[impv]

\end{description}


\clearpage
