\subsection{Feature List}
\begin{description}
\item[Type:] FeatureList
\item[Description:]

This is a table describing any of the (un)parameterized methods for
discovering and characterizing the emission in a cube.  Examples of
such procedures are:

- ClumpFind \\
- GaussClump \\
- Dendrogram (binary, as well as non-binary) \\

The output BDP could be an image cube matched in axes to the input LineCube,
where the intensity values are enumerated from a membership of Features. But
not all procedures use this intermediary format.

The final output from this is a table of features in the input LineCube.

Here's an example described as Gaussian clumps, e.g. from a moment
of inertia description of the Features
\begin{verbatim}
X
Y
V
sX
sY
sV
PA        (if sX != sY, a position angle describes the orientation)
dVdP      (any residual correlation can be described as velocity gradient)
Peak      
Sum       normally in Jy.km/s
\end{verbatim}

\item[Constituents:] BDP\_Table+BDP\_Image

\item[ADMIT Task:] AT\_FeatureList

\item[CASA Task(s):] TBD

\item[Input BDP(s):] LineCube

\item[Output BDP(s):]  Table of feature properties (position, line width,
 peak temperature, number of pixels, etc).  In the case of dendrogram,
 there may also be a JPG image of the dendrogram tree.
\end{description}
