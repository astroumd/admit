% now in RST, don't edit here

\subsection{CubeSpectrum\_AT}

\subsubsection{Description}

CubeSpectrum\_AT will compute one (or more) typical spectra
from a spectral line cube.
They are stored in a CubeSpectrum\_BDP, which also contains 
a graph of intensity versus frequency.
For certain types of cubes,
these can be effectively used in LineID\_AT to identify
spectral lines. 

The selection of which point will be used for the spectrum is of course
subject to user input, but some automated options can be given.
For example, the brightest point in the cube
can be used. The reference pixel can be used. The size around this point
can also be choosen, but this is not a recommended procedure, as it can
affect the LineID\_AT procedure, since it will smooth out (but increase S/N)
the spectrum if large velocity gradients are present. If more S/N is needed, 
CubeStats\_AT or PVCorr\_AT can be used.

Note in an earlier version, the SpectralMap\_AT would allow multiple
points to be used. We deprecated this, in favor of allowing this AT
to create multiple points instead of just one. Having multiple spectra,
much like what is done in PVCorr\_AT,  cross-correlations could be made
to increase the line detection success rate.

This AT used to be called PeakSpectrum in an earlier revision.

\subsubsection{Use Case}

This task is probably one of the first to run after ingestion, and will
quickly be able to produce a representative spectrum through the cube,
giving the user ideas how to process these data.  If spectra are taken 
through multiple points, a diagram can be combined with a CubeSum 
centered around a few spectra.


\subsubsection{Input BDPs}

\begin{description}

\item[SpwCube\_BDP] The cube through from which the spectra are drawn. Required.
Special values {\bf xpeak,ypeak} are taken from this cube.

\item[CubeSum\_BDP] An optional BDP describing the image that represents the sum
of all emission. The peak(s) in this map can be selected for the spectra drawn.
This can also be a moment-0 map from a LineCube.

\item[FeatureList\_BDP] An optional BDP describing features in an image or cube,
from which the RA,DEC positions can be used to draw the spectra.


\end{description}


\subsubsection{Input Keywords}

\begin{description}

\item[points] One (or more) points where the spectrum is computed.
Special names are for special points:  {\bf xpeak,ypeak} are for using 
the position where the peak in the cube occurs.  {\bf xref,yref} are the
reference position (CRPIX in FITS parlor).

\item[size] Size of region to sample over. Pixels/Arcsec.  Square/Round.  
By default a single pixel is used. No CASA regions allowed here, keep
it simple for now.

\item[smooth] Some smoothing option applied to the spectrum. Use with
caution if you want to use this BDP for LineID.

\end{description}

\subsubsection{Output BDPs}

\begin{description}

\item[CubeSpectrum\_BDP] A table containing one or more spectra. For a single
spectrum the intensity vs. frequency is graphically saved. For multiple
spectra (the original intent of the deprecated SpectralMap\_AT) it should
combine the representation of a CubeSum\_BDP with those of the Spectra
around it, with lines drawn to the points where the spectra were taken.

\end{description}


\subsubsection{Procedure}

After making a selection through which point the spectrum is taken,
grabbing the values is straightforward. For example, the imval task
in CASA will do this.

\subsubsection{CASA tasks used}

\begin{description}

\item[imval] to extract a spectrum around a given position 

\end{description}

\clearpage
