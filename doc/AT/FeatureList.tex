% in RST now, don't edit here.

\subsection{FeatureList\_AT}

\subsubsection{Description}

FeatureList\_AT takes a LineCube, and describes the three-dimensional structure
of the emission (and/or absorbtion)

% should we allow 2D maps as well?

\subsubsection{Use Case}

\subsubsection{Input BDPs:}

Depending on which BDP(s) is/are given, and a choice of keywords, line
identification can take place.  For example, one can use both a
CubeSpectrum and CubeStats and use a conservative AND or a more
liberal OR where either both or any have ``detected'' a line.

\begin{description}
\item[CubeSpectrum] just a spectrum through a selected point. For some cubes this is perfectly ok.
\item[CubeStats] Peak/RMS. Since this table analyzed each plane of a cube, it will more likely pick 
up weaker lines, which a CubeSpectrum will miss.
\item[PVCorr] an autocorrelation table from a PVSlice\_AT. This has the potential of detecting even
weaker lines, but its creation via PVSlice\_AT is a fine tuneable process.
%\item[SpectralMap]  this may actually be absorbed in CubeSpectrum, where there is more than one spectrum, where every spectrum is tied to a location and size over which the spectrum is computed.
\end{description}


\subsubsection{Input keywords}

\begin{description}
\item[vlsr] If VLSR is not know, it can be specified here. Currently must be known, either via
this keyword, or via ADMIT (header). % is that in Summary_BDP ?   And what about VLSR vs. z
\item[pmin]  Minimum likelyhood needed for line detection. This is a number between 0 and 1.
[0.90]
\item[minchan] Minimum number of continguous channels that need to contain a signal, 
to combine into a line. Note this means that each channel much exceed {\bf pmin}.
[5]
\end{description}

\subsubsection{Output BDPs}

\begin{description}
\item[LineList\_BDP] A single LineList is produced, which is typically used by LineCube\_AT
to cut a large spectral window cube into smaller LineCube's.
\end{description}

% why would you need multiple linelists, doesn't make sense. Only one goes into the next stage
% the LineCube_AT
%The AT will produce a single LineList\_BDP (with the exception of a SpectralMap as input, in this case a LineList\_BDP for each spectrum will be created). The items of the BDP will be a table with the following columns:

% No need to describe the BDP here, that's done in the BDP listings.
%\begin{description}
%\item[frequency] in GHz, precise to 5 digits
%\item[unique identifier] ANAME-FFF.FFF ; e.g.  CO-115.27,   N2HP-93.173,   U-98.76
%\item[likelihood] a number scaled 0...1 of detection probability, with 1 being most probable
%\item[full name] CO, CO\_v1, C2H, H13CO+
%\item[transition] Quantum numbers
%\item[velocity] v$_{lsr}$ or offset velocity (based on rest frequency)
%\item[energies] lower and upper state energies in K
%\item[line strength] spectral line strength in D$^2$
%\item[peak] the peak intensity of the line in Jy/bm
%\item[fwhm] the FWHM of the line in km/s
%\end{description}
%%Additionally an item to denote whether the velocity is v$_{lsr}$ or offset velocity. The offset velocity will only be used when the v$_{lsr}$ is not known and cannot be determined. The coordinates of where the line list applies along with units will also be stored.

\subsubsection{Line List Database}

The line list database will be the splatalogue subset that is already included with the CASA distribution.
This one can be used offline, e.g. via the slsearch() call in CASA.
Currently astropy's {\tt astroquery.splatalogue} module has a {\tt query\_lines()} function to query the
full online WEB interface.

Once LineID\_AT is enabled to attempt actual line identifications, one of those methods will
be selected, together with possible selections based on the astrophysical source. This will
likely need a few more keywords.


\subsubsection{Procedure}

A few words about likelyhoods.   Each procedure that creates a table is either supposed to deliver 
a probability, or LineID\_AT must be able to compute it.   Let's say for a computed RMS noise
$\sigma$, a $3\sigma$ peak in a particular channel would have a likelyhood
of 0.991 (or whatever value) for CubeStats .

\subsubsection{CASA tasks used}

There are currently no tasks in CASA that can handle this. 

\clearpage
